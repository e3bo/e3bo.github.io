\documentclass[a4paper]{article}
\usepackage[margin=1in,noheadfoot]{geometry}
\usepackage{multicol}
\usepackage{hyperref}
\usepackage{verbatim}
\usepackage{fancyhdr}
\pagestyle{fancy}

\rhead{\today}
\lhead{Eamon B. O'Dea}
\cfoot{\thepage}

\begin{document}
\begin{center}
\textsc{Curriculum Vitae}
\end{center}

\begin{multicols}{2}
  \begin{itemize}
  \item[~] Odum School of Ecology\\Univsersity of Georgia\\Athens, GA 30602-2202 USA    
  \item[~] E-mail: ebodea@uga.edu\\Skype: eamon.b.odea\\Website: \url{http://ebodea.name}
\end{itemize}
\end{multicols}

\paragraph{Degrees:}
\begin{itemize}
  \item[~] University of Texas at Austin; PhD in Ecology, Evolution and Behavior; 2013. \\
    Supervising professors: Lauren Meyers and Claus Wilke.\\
    Thesis: Analyses of infectious disease data with attention
    to heterogeneity.
  \item[~] State University of New York at Geneseo, BS in Biology, 2008.
\end{itemize} 

\paragraph{Employment:}
\begin{itemize}
  \item[~] Postdoctoral Research Scholar; University of Georgia;
    Athens, GA; 2015--present
  \item[~] Postdoctoral Fellow; Georgetown University; Washington, DC;
    2013--2015
\end{itemize}

\paragraph{Peer-reviewed papers:}
\begin{itemize}
  \item[~] J.\ M.\ Drake, T.\ S.\ Brett, M.\ J.\ Ferrari,
    \'{E}.\ Marty, P.\ B.\ Miller, E.\ B.\ O'Dea, S.\ Chen,
    B.\ I.\ Epureanu, A.\ W.\ Park, S.\ M. O'Regan, P.\ Rohani. The
    statistics of epidemic transitions. PLoS Computational
    Biology. 8 May 2019.
  \item[~] S.\ Chen, E.\ B.\ O'Dea, J.\ M.\ Drake, and
    B.\ Epureanu. Eigenvalues of the covariance matrix as early
    warning signals for critical transitions in ecological
    systems. Scientific Reports. 22 Februrary 2019.
  \item[~] E.\ B.\ O'Dea and J.\ M.\ Drake.  Disentangling reporting
    and disease transmission. Theoretical Ecology. 22 August 2018.
  \item[~] E.\ B.\ O'Dea, A.\ W.\ Park, and J.\ M.\ Drake.  Estimating
    the distance to an epidemic threshold. Journal of the Royal
    Society Interface. 27 June 2018.
  \item[~] T.\ S.\ Brett, E.\ B.\ O'Dea, \'{E}.\ Marty,
    P.\ B.\ Miller, A.\ W.\ Park, J.\ M.\ Drake, and
    P.\ Rohani. Anticipating epidemic transitions with imperfect
    data. PLoS Computational Biology. 8 June 2018.
  \item[~] P.\ B.\ Miller, E.\ B.\ O'Dea, P.\ Rohani, and
    J.\ M.\ Drake. Forecasting infectious disease emergence subject to
    seasonal forcing. Theoretical Biology and Medical Modelling. 6
    September 2017.
  \item[~] C.\ J.\ Dibble, E.\ B.\ O'Dea, A.\ W.\ Park, and
    J.\ M.\ Drake. Waiting time to infectious disease
    emergence. Journal of the Royal Society Interface. 19 October 2016.
  \item[~] E.\ B.\ O'Dea, H.\ Snelson, and S.\ Bansal. Using
    heterogeneity in the population structure of U.S. swine farms to
    compare transmission models for porcine epidemic
    diarrhoea. Scientific Reports. 7 March 2016.
  \item[~] E.\ B.\ O'Dea, K.\ M.\ Pepin, B.\ A.\ Lopman, and
    C.\ O.\ Wilke. Fitting outbreak models to data from many small
    norovirus outbreaks. Epidemics. March 2014.
  \item[~] E.\ B.\ O'Dea and C.\ O.\ Wilke. Contact networks and
    phylodynamics: How host contact networks shape parasite
    evolutionary trees. Interdisciplinary Perspectives in Infectious
    Disease. 2011.
  \item[~] E.\ B.\ O'Dea, T.\ E.\ Keller, and C.\ O.\ Wilke. Does
    mutational robustness inhibit extinction by lethal mutagenesis in
    viral populations? PLoS Computational Biology. 10 June 2010.
\end{itemize}

\paragraph{Teaching experience:}

\begin{itemize}
  \item[~] Guest lecturer on near-critical dynamics for ECOL 8520. UGA. Athens, GA. 25 January 2017.
  \item[~] Teaching assistant for Ecology, Genetics, and Introduction to Computational Biology.
\end{itemize}

\paragraph{Software developed:}
\begin{itemize}
  \item[~] E.\ B.\ O'Dea. spaero: Software for Project
    AERO. \url{https://cran.r-project.org/package=spaero}
\end{itemize}

\paragraph{Talks contributed:}
\begin{itemize}
\item[~] E.\ B.\ O'Dea, T.\ S.\ Brett, J.\ M.\ Drake, A.\ W.\ Park,
  \'{E}.\ Marty, P.\ B.\ Miller, P.\ Rohani. Pros and cons of
  slowing-down based indicators of infectious disease emergence and
  eradication. MIDAS Network Meeting. Bethesda, MD. 4 April 2018.
\item[~] E.\ B.\ O'Dea and J.\ M.\ Drake. Disentangling reporting and
  disease transmission using second order statistics. Society for
  Industrial and Applied Mathematics Southeastern Atlantic Sectional
  Conference. UNC. Chapel Hill, NC. 11 March 2018.
\item[~] E.\ B.\ O'Dea. Identifying Correlates of Pathogen Spatial
  Spread from Phylogenetic Trees. Center for the Ecology of Infectious
  Diseases. UGA. Athens, GA. 7 December 2016.
\item[~] E.\ B.\ O'Dea. An R-based approach to global sensitivity
  anlysis for moments of stochastic models. Computational Ecology
  \& Epidemiology Study Group. UGA. Athens, GA. 3 February 2016.
\item[~] E.\ B.\ O'Dea. Introduction to Docker. Computational Ecology
  \& Epidemiology Study Group. UGA. Athens, GA. 23 September 2015.
\item[~] E.\ B.\ O'Dea. Investigating the transmission pathways of
  porcine epidemic diarrhea virus (PEDV) using outbreak incidence and
  virus sequence data. USDA Veterinary Services Simulation and
  Modeling Seminar Series. Webinar. 20 January 2015.
\item[~] E.\ B.\ O'Dea and S.\ Bansal. Spreading patterns in the
  ongoing U.S. Porcine Epidemic Diarrhea Virus outbreak. Friday
  Biology Seminar at Georgetown University. Washington, DC. 15
  November 2013.
\item[~] E.\ B.\ O'Dea and C.\ Leary. Degree-correlated scale-free
  networks and epidemics.  Undergraduate Biomathematics Day, Niagara
  Falls, NY. April, 2008. SUNY Geneseo GREAT Day. Geneseo, NY. April
  2008.
\end{itemize}

\paragraph{Posters contributed:}
\begin{itemize}
  \item[~] E.\ B.\ O'Dea, A.\ W.\ Park, and J.\ M.\ Drake. Estimating the distance to
    an epidemic threshold. EEID. UC Santa Barbara, CA. June  2017.
  \item[~] E.\ B.\ O'Dea and J.\ M.\ Drake. Estimating the distance to
    the epidemic threshold. MIDAS Network Meeting.  Reston,
    VA. May 2016.
  \item[~] E.\ B.\ O'Dea and S.\ Bansal. Learning patterns of
    transmission from the U.S. PEDV outbreak. Epidemics.
    Amsterdam, The Netherlands. November 2013.
  \item[~] E.\ B.\ O'Dea and C.\ Leary. Epidemic dynamics on
    randomized scale-free networks.  The Joint Mathematics Meeting
    of the MAA and AMS. San Diego, CA. January 2008.
\end{itemize}

\paragraph{Undergraduate mentoring assignments:}

\begin{itemize}
  \item[~] David Schaffer. Localization of the source of an outbreak
    on weighted and directed networks. Summer 2014.
  \item[~] Sarah Kramer. Estimating effects of risk compensation due
    to antiviral HIV therapy. Spring 2014.
  \item[~] Yongjun Cho. The impact of population immunity on antigenic
    drift during large epidemics and small outbreaks. Journal of the
    Young Investigator. Volume 20. Issue 3. September 2010.
\end{itemize}

\paragraph{Journals for which I have been a reviewer:}
\begin{itemize}
  \item[~] BMC Evolutionary Biology.
  \item[~] BMC Public Health.
  \item[~] Ecosphere.
  \item[~] Genetics.
  \item[~] Journal of the Royal Society Interface.
  \item[~] JSTAT.
  \item[~] PeerJ.
  \item[~] PLoS Neglected Tropical Diseases.
  \item[~] PLoS One.
  \item[~] PLoS Pathogens.
  \item[~] PNAS.
  \item[~] Scientific Reports.
\end{itemize}

\paragraph{Research interests:}
\begin{itemize}
\item[~] applied statistics, complex networks, disease dynamics, phylogenetics
\end{itemize}

\paragraph{Honors and awards:}
\begin{itemize}
  \item[~] Phi Beta Kappa.
  \item[~] SUNY Geneseo College Honors Program.
\end{itemize}

\end{document}
